\chapter{Related Literature}

\section{Foreign Literature}

Online Classes has more cheating tendencies than the face-to-face classes \cite{watson2010cheating}.
In the research conducted by \citeA{king2009online}, 73\% of 121 undergraduate students surveyed felt it was easier to cheat online compared to a traditional face-to-face classroom.
Similarly, \citeA{berkey2015cheating} reported that 84\% of the students surveyed in their study agreed that student dishonesty in online test-taking was a significant issue.
In a study of 635 students, \citeA{watson2010cheating} also noted that students indicated that they would be more than four times more likely to cheat in an online class.

Unproctored online examination is more prone to academic dishonesty.
The most commonly reported challenge in online assessment is how to maintain academic integrity \cite{hollister2009proctored}, especially with unproctored assessments \cite{arnold2016cheating}.
\citeA{KERESZTURY20131516} evaluated cheating methods in classic exams which they classified into three categories: information exchange among students, using forbidden materials, and circumventing the assessment process.
On one large-scale study of cheating in online courses and work tasks found that between 26\% and 34\% of students cheated by looking up answers online, as did 20\% of contract employees \cite{corrigan2015deterring}.
Online collusion during asynchronous unproctored exams was detected among engineering students \cite{de2015calculated}, and over 1,230 massive open online course students copied answers during asynchronous unproctored certificate exams using multiple online accounts \cite{northcutt2016detecting}.

\citeA{moten2013examining} explained that students in these learning environment work independently with relative autonomy and anonymity, and instructors may be uncertain who is taking exams or how best to validate student learning.
In the research addressing student validation based on face recognition for online learning \cite{labayen2014smowl}, it is stated that without certainty of the authenticity of the online learner’s identity, the aspiration towards fully online education is stymied and the evaluation of the knowledge and skills obtained by the online learner is unreliable.

\section{Local Studies}

An open ended questionnaire was distributed to 52 students of the University of the Philippines Open University enrolled in a Computer Ethics course at the graduate level where the course, including the final exam, is fully online \cite{ravasco2012technology}.

The first questioned asked was where does student cheat more frequent, and 21 out of 52 which is 40.38\% of the class are convinced that there is more cheating or more possibility of it happening in an online classroom.
One reason a student has stated was: “I believe cheating is more frequent in an online distance course because information is readily available and accessible. Once a student goes online, hundreds and hundreds of answers for assignments, quizzes, or tests can be acquired.”

The students then asked to offer ways online cheating could be done or is already being done based on their readings, their experiences, and from their own observation.
73\% (38/52) mentioned identity impersonation, substitution, or proxy attendance as the most common and perhaps the easiest to get away with.
Here an online student gets another to do the academic work requirement for him or her.
69\% (36/52) named the search engine and plagiarism duo as a very common way of cheating.
Students would “google” terms, ideas, concepts and copy-paste their desired portions on their work.
69\% (36/52) referred to unauthorized intellectual networking as another method of cheating.
Students would collaborate dishonestly and even share files in the process.
This involves discussing answers with each other using forums or even personal chat rooms.

\section{Greek}

Students use many cheating techniques when taking examinations.
\citeA{FAUCHER200937} demonstrated cheating via giving, receiving and handling information, and circumventing in the exam.
Also, they presented some methods to detect and prevent cheating; moreover, the educational program's academic integrity needs to be maintained by using all resources available to develop effective policies and procedures.

\citeA{KERESZTURY20131516} evaluated cheating methods in classic exams which they classified into three categories: information exchange among students, using forbidden materials, and circumventing the assessment process.
However, new kinds of cheating appeared, such as using information stored in a storage unit.

Methods of cheating have become ever more developed and hard to detect.
\citeA{CURRAN20112155} highlight traditional methods of cheating: hiding notes, pencil case, writing on arms/hands and leaving the room.
However, holding large amounts of information can be replaced by new technologies, for example, using Mobile Phones, Calculators, MP3 Players, wireless receivers and Personal Digital Assistant (PDAs).
Besides, they present technically feasible solutions that prevent the cheating process using signal jamming devices to identify mobile phones that are active and block communication among them.

\citeA{gao2012online} summed the commonly used methods to prevent students from taking online exams from e-cheating as follows.

\begin{enumerate}
    \item Setting up time limitation.
    \item Setting up quizzes.
    \item Exams consist of randomly selected questions from a vast question pool so that each student will have a different exam/test.
    \item Comparing the IP addresses to see if two students are in the vicinity of each other.
    \item Using biometrics to reduce the possibility of E-cheating and to authenticate remote students.
\end{enumerate}

Commonly biometrics includes keystroke, voice, signature, face, iris and fingerprint.
Besides, he showed two commercially available products, which can be used to guarantee secure exams: Webassessor and ProctorU; that also has been tested via some universities and can be used to proctor exams.

To perform online exams on student expands the likelihood of cheating through a cut, copy and paste of information to/from the testing environment, Screen capture and printing functions, Searching and surfing the Web, HTML source code seeing, send messaging, screen sharing.
\citeA{frankl2012secure} gave "Secure Exam Environment" (SEE) implemented at the Alpen-Adria-Universität Klagenfurt (AAUK) to be held on student portable pcs without access to local files and resource, for example, the Internet.

The web and the "anytime, anywhere "get to give by PDAs, put almost endless information at our fingertips.
Numerous students have discovered creative approaches to utilize innovation to cheat during exams.
\citeA{kelley2014technology} highlight some of the most spread high tech cheating techniques such as smartphones.
Text messaging answers back and forth with other test-takers.
Taking photos of the test with a phone and sending it to the second party to copy or help the first student.
Storing data on graphics calculators can also be easily accomplished and recovered amid exams without the instructor realizing the student is cheating.
Small mall micro-cameras and very tiny hearing aids allow a second party to view the exam, gaze the answer upward in a reference book, and transfer the answers to the exam taker.

As of late, it turned into the spread of college cheating than reasons why students cheat.
\citeA{simkin2010college} talked about this issue in more depth.
They applied the hypothesis to the Theory of Reasoned Action (TRA) to expound cheating behaviour and identify what factors motivate students to cheat.
Three aspects of cheating motivators: access to online resources; a desire to succeed, and there is not existent punishments when a few instructors force for infractions.
They displayed some of the illustrations that using in cheating process, Text messaging to send test answers amid examinations, utilizing PDAs to take pictures and email test materials to others.
They find that cheating is much more regular among business understudies than among non-business understudies.

\citeA{raines2011cheating} focused on the students' definition of cheating in the online learning environment.
However, they collected and analyzed for evidence of common words that give meaning to the purpose of cheating.
First, 60\% of the students defined the cheating by breaking the principles, dishonesty and not using your own brain.
Violating the exam regulations (expressed or implied), To get the answers by deceiving the teacher, storing answers on the memory of a calculator, and submitting answers that are not of your own creation, for example...
Secondly, 39\% of the students referred of cheating via focusing on the tangible outcomes of cheating, such that getting information by non-ethical means to pass an exam, and taking advantage of information or resources, known only by the cheater, to improve their grade.
Finally, 3\% of students were not able or willing to define cheating.

Cheating is clearly wrong, arguments against it, which it provides an unfair advantage, and obstructs learning.
The wrongness of cheating should be an ethical, not a bureaucratic question.
\citeA{bouville2010cheating} discussed the relationship between cheating and grades: cheaters get undeservedly high grades and an unfair advantage over other students.
This may mean that the grade is an infallible evaluation of how good a student is so that if grades are low, it can only be because the student does not work enough.

Likewise, grades are a proxy for what students know and can do, which is in turn used as a proxy for what students may be able to do in the future.
